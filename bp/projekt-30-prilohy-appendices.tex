\chapter{Ukázka telemetrických dat}
\label{data}
V této kapitole si ukážeme program, kde nejprve zachytíme telemetrická data pomocí automatického sběru. Poté ho přepíšeme pro manuální sběr a zachytíme telemetrická data.

Aplikace představuje server s jedním koncovým bodem \texttt{/hello\textunderscore world}, který po obdržení HTTP GET požadavku, odešle odpověď s pozdravem. 
\section*{Automatický sběr dat}
\label{auto}

\subsection*{Zdrojový kód}
\begin{listing}[H]
    \begin{minted}{python}
from flask import Flask, request

app = Flask(__name__)


@app.route("/hello_world")
def server_request():
    print(request.args.get("param"))
    return "Hi"


if __name__ == "__main__":
    app.run(port=8080)
\end{minted}
    \caption{Ukázka kódu pro automatický sběr dat}
    \label{lst:exampleCode}
\end{listing}

\subsection*{Vygenerovaná data}

\begin{listing}[H]
    \begin{minted}{json}
{
    "name": "/hello_world",
    "context": {
        "trace_id": "0x52febf311696be92218395ff71e5222d",
        "span_id": "0x2b17fe881b521fba",
        "trace_state": "[]"
    },
    "kind": "SpanKind.SERVER",
    "parent_id": "0x82603ff759ca869a",
    "start_time": "2023-01-15T10:59:44.449634Z",
    "end_time": "2023-01-15T10:59:44.451575Z",
    "status": {
        "status_code": "UNSET"
    },
    "attributes": {
        "http.method": "GET",
        "http.server_name": "127.0.0.1",
        "http.scheme": "http",
        "net.host.port": 8080,
        "http.host": "localhost:8080",
        "http.target": "/hello_world?param=Hello",
        "net.peer.ip": "127.0.0.1",
        "http.user_agent": "python-requests/2.28.2",
        "net.peer.port": 41678,
        "http.flavor": "1.1",
        "http.route": "/hello_world",
        "http.status_code": 200
    },
    "events": [],
    "links": [],
    "resource": {
        "attributes": {
            "telemetry.sdk.language": "python",
            "telemetry.sdk.name": "opentelemetry",
            "telemetry.sdk.version": "1.15.0",
            "telemetry.auto.version": "0.36b0",
            "service.name": "unknown_service"
        },
        "schema_url": ""
    }
}
\end{minted}
    \caption{Výsledná data automatického sběru dat}
    \label{lst:exampleCodeResult}
\end{listing}



\section*{Manuální sběr dat}
\label{manual}

\subsection*{Zdrojový kód}
\begin{listing}[H]
    \begin{minted}{python}
from flask import Flask, request
from opentelemetry import trace
from opentelemetry.instrumentation.wsgi import collect_request_attributes
from opentelemetry.propagate import extract
from opentelemetry.sdk.trace import TracerProvider
from opentelemetry.sdk.trace.export import (
    BatchSpanProcessor,
    ConsoleSpanExporter,
)

app = Flask(__name__)

trace.set_tracer_provider(TracerProvider())
tracer = trace.get_tracer_provider().get_tracer(__name__)

trace.get_tracer_provider().add_span_processor(
    BatchSpanProcessor(ConsoleSpanExporter())
)


@app.route("/hello_world")
def server_request():
    with tracer.start_as_current_span(
        "hello_world",
        context=extract(request.headers),
        kind=trace.SpanKind.SERVER,
        attributes=collect_request_attributes(request.environ),
    ):
        print(request.args.get("param"))
        return "Hi"


if __name__ == "__main__":
    app.run(port=8080)
\end{minted}
    \caption{Ukázka kódu pro manuální sběr dat}
    \label{lst:exampleCode1}
\end{listing}

\subsection*{Vygenerovaná data}

\begin{listing}[H]
    \begin{minted}{json}
{
    "name": "hello_world",
    "context": {
        "trace_id": "0x9ebd0f920174b3484227b18f08a246d4",
        "span_id": "0x58534f4ca1ee43b0",
        "trace_state": "[]"
    },
    "kind": "SpanKind.SERVER",
    "parent_id": "0x9f2996581d1b9d34",
    "start_time": "2023-01-15T10:57:43.432127Z",
    "end_time": "2023-01-15T10:57:43.432179Z",
    "status": {
        "status_code": "UNSET"
    },
    "attributes": {
        "http.method": "GET",
        "http.server_name": "127.0.0.1",
        "http.scheme": "http",
        "net.host.port": 8080,
        "http.host": "localhost:8080",
        "http.target": "/hello_world?param=Hello",
        "net.peer.ip": "127.0.0.1",
        "http.user_agent": "python-requests/2.28.2",
        "net.peer.port": 36652,
        "http.flavor": "1.1"
    },
    "events": [],
    "links": [],
    "resource": {
        "attributes": {
            "telemetry.sdk.language": "python",
            "telemetry.sdk.name": "opentelemetry",
            "telemetry.sdk.version": "1.15.0",
            "service.name": "unknown_service"
        },
        "schema_url": ""
    }
}
\end{minted}
    \caption{Výsledná data manuálního sběru dat}
    \label{lst:exampleCodeResult1}
\end{listing}


\chapter{Obsah přiloženého paměťového média}

Přiložené paměťové médium obsahuje následující adresářovou strukturu:


\dirtree{%
.1 /.
.2 bp/.\DTcomment{Složka se soubory pro překlad textu bakalářské práce}.
.2 experiments/.\DTcomment{Složka s experimenty}.
.3 01/.
.4 01.sh.\DTcomment{Skript s experimentem 1}.
.3 03/.
.4 03.py.\DTcomment{Skript s experimentem 3}.
.3 04/.
.4 04.py.\DTcomment{Skript s experimentem 4}.
.2 opentelemetry-demo/.
.3 src/.\DTcomment{Složka se zdrojovými kódy mikroslužeb}.
.3 .env.\DTcomment{Konfigurační soubor}.
.3 docker-compose.yml.\DTcomment{Compose soubor s kontejnery}.
.2 otel-collector/.
.3 src/.\DTcomment{Složka se soubory pro kolektor a monitorovací nástroje}.
.3 .env.\DTcomment{Konfigurační soubor}.
.3 docker-compose.yml.\DTcomment{Compose soubor s kontejnery}.
.2 captureData.sh.\DTcomment{Skript pro sběr paketů v kapitole 3.2}.
.2 packetsOTLP.pcap.\DTcomment{Soubor s pakety protokolu OTLP}.
.2 README.md.\DTcomment{Soubor popisující práci s testovacím prostředím}.
.2 thesis.pdf.\DTcomment{Bakalářská práce ve formátu PDF}.
}